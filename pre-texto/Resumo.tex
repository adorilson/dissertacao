\begin{resumo}

Dispositivos móveis estão ficando cada vez mais populares e acessíveis
para pessoas de diferentes poder aquisitivo ao redor do mundo. A
plataforma Android é atualmente a mais popular para o desenvolvimento
de aplicações móveis, ocupando mais de 80\% do mercado de sistemas
operacionais para aplicações de dispositivos móveis. Tal realidade
cria uma demanda por customizações de aplicações para lidar com
diferentes dispositivos, tais como tamanho de tela, bibliotecas de
classes (API - Application Programming Interface), disponibilidade de
poder de processamento e memória disponível, idiomas e necessidades
específicas dos usuários. Apesar da popularidade e ampla gama de
aplicações Android já desenvolvida, existe uma carência de estudos que
investiguem a forma como evoluções na API Android vem
sendo gerências pelas aplicações.

Esta dissertação de mestrado tem como objetivo principal analisar,
caracterizar e comparar técnicas de implementacao de variabilidades
utilizadas por aplicações Android para prover suporte às diferentes
versões da API. Em especial, o trabalho busca: (i)
identificar e analisar técnicas de implementação de variabilidades
relacionadas à evolução de APIs da plataforma utilizadas por
aplicações Android; (ii) identificar quais as evoluções na API que mais
impactam o desenvolvimento das aplicações; (iii) identificar qual a incidência de
de código-morto nas aplicações relacionados a versão da API e; (iv) identificar o
esforço necessário para as aplicações oferecerem suporte a um maior número de versões
da API Android.

\textbf{Palavras-chave}: gerência de variabilidades, implementação de variabilidades,
aplicações Android, plataforma de computação móvel, estudos empíricos, API
\end{resumo}

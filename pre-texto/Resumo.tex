\begin{resumo}

Dispositivos móveis estão ficando cada vez mais populares e acessíveis
para pessoas de diferentes poder aquisitivo ao redor do mundo. A
plataforma Android é atualmente a mais popular para o desenvolvimento
de aplicações móveis, ocupando mais de 80\% do mercado de sistemas
operacionais para aplicações de dispositivos móveis. Tal realidade
cria uma demanda por customizações de aplicações para lidar com
diferentes dispositivos, tais como tamanho de tela, bibliotecas de
classes (API - Application Programming Interface), disponibilidade de
poder de processamento e memória disponível, idiomas e necessidades
específicas dos usuários. Apesar da popularidade e ampla gama de
aplicações Android já desenvolvida, existe uma carência de estudos que
investiguem a forma como variabilidades de aplicações Android vem
sendo implementadas e evoluídas.

Esta dissertação de mestrado tem como objetivo principal analisar,
caracterizar e comparar técnicas de implementacao de variabilidades
utilizadas por aplicações Android. Em especial, o trabalho busca: (i)
identificar e analisar técnicas de implementação de variabilidades
relacionadas à evolução de APIs da plataforma utilizadas por
aplicações Android; (ii) iIdentificar e analisar as técnicas de
implementação de variabilidades relacionadas ao suporte à dispositivos
com diferentes tamanhos de tela utilizadas por aplicações Android; e
(ii) identificar as principais ferramentas utilizadas para derivação
automática de produtos utilizadas por aplicações Android.

\textbf{Palavras-chave}: gerência de variabilidades, implementação de variabilidades,
aplicações Android, plataforma de computação móvel, estudos empíricos
\end{resumo}

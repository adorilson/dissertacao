% Resumo em língua vernácula

\begin{center}
	\Large{\textsc{\textbf{Resumo}}}
\end{center}

\noindent A plataforma Android é atualmente a mais popular para o desenvolvimento
de aplicações móveis, ocupando mais de 80\% do mercado de sistemas operacionais para
dispositivos móveis, criando uma demanda por customizações de aplicações para lidar
com diferentes dispositivos, tais como, tamanho de tela, poder de processamento e
memória disponível, idiomas e necessidades específicas dos usuários. Já foram
disponibilizadas 23 novas versões da plataforma Android desde o seu lançamento. De
forma a permitir a execução com sucesso das aplicações em diferentes dispositivos,
é fundamental oferecer suporte à múltiplas versões da API (Application Programming
Interface). Esta dissertação de mestrado tem como objetivo principal: analisar,
caracterizar e comparar técnicas utilizadas por aplicações Android para oferecer
suporte a múltiplas versões da API. Em especial, o trabalho busca: (i) identificar
na literatura quais as técnicas indicadas para suporte à múltiplas versões da API
Android; (ii) analisar aplicações reais para quantificar o uso dessas técnicas
indicadas; e (iii) comparar as características e consequências do uso de tais
técnicas. Um estudo empírico foi conduzido para atingir tal objetivo, no qual
foram analisadas 25 aplicações Android populares. Os resultados do estudo mostram
que existem três técnicas para prover suporte a múltiplas versões da API: i) pacote
de compatibilidade, variabilidades de granularidade grossa da API que envolvam um
conjunto de classes; ii) re-implementação de recurso, para situações pontuais e
granularidade grossa em nível de classe ou quando o recurso não está disponível
em pacote de compatibilidade; e iii) uso explícito da nova API, variabilidades de
granularidade fina da API que envolva a chamada de métodos específicos. Através da
análise de 25 aplicações identificamos que pacote de compatibilidade foi utilizada
por 23 aplicações, re-implementação de recurso por 14 e uso explícito da nova API
por 22. A API de fragmentos contêm os elementos mais comuns dentre os lançados em
versões superiores da plataforma que são usados pelas aplicações durante sua evolução,
sendo referenciados por 68\% delas. No geral, as aplicações poderiam aumentar o seu
mercado em potencial com adaptações de, em média, 15 trechos de código, por outro
lado, os desenvolvedores das aplicações têm se preocupado em evitar código-morto em
função da API da plataforma. Na análise de 7 aplicações, 4 delas continham código-morto,
mas os quais em geral não representam mais do que 0,1\% do seu código total.

\noindent\textbf{Palavras-chave}: gerência de variabilidades, implementação de
variabilidades, aplicações Android, plataforma de computação móvel, estudos empíricos,
suporte multi-versão de API
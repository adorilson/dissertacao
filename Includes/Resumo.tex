% Resumo em língua vernácula
\begin{center}
	{\Large{\textbf{\mscThesisTitle}}}
\end{center}

\vspace{1cm}

\begin{flushright}
	Autor: \author\\
	Orientador(a): \advisor \\
	
	\ifcoadvisor
	Coorientador(a): \coadvisor
	\fi
	
	 
\end{flushright}

\vspace{1cm}

\begin{center}
	\Large{\textsc{\textbf{Resumo}}}
\end{center}

\noindent Dispositivos móveis estão ficando cada vez mais populares e acessíveis para pessoas
de diferentes poder aquisitivo ao redor do mundo. A plataforma Android é atualmente
a mais popular para o desenvolvimento de aplicações móveis, ocupando mais de 80\%
do mercado de sistemas operacionais para aplicações de dispositivos móveis. Tal
realidade cria uma demanda por customizações de aplicações para lidar com diferentes
dispositivos, tais como tamanho de tela disponibilidade de poder de processamento
e memória disponível, idiomas e necessidades específicas dos usuários. Além disso,
a bibliotecas de classes da plataforma (API - Application Programming Interface)
evolui muito rapidamente. Já foram disponibilizadas 23 versões desde o lançamento. 
Portanto, é fundamental para o sucesso das aplicações o suporte a múltiplas versões
da API. Apesar da popularidade e ampla gama de aplicações Android já desenvolvida,
existe uma carência de estudos que investiguem a forma como aplicações Android lidam
com essa variabilidade.

\noindent Esta dissertação de mestrado tem como objetivo principal analisar, caracterizar e
comparar técnicas utilizadas por aplicações Android para suporte a múltiplas versões
da API. Em especial, o trabalho busca:
(i) Identificar na literatura quais as técnicas indicadas para suporte a múltiplas versões da API Android;
(ii) Analisar aplicações reais para quantificar o uso dessas técnicas indicadas; e
(ii) Comparar as características e consequências do uso de tais técnicas.

\noindent\textit{Palavras-chave}: gerência de variabilidades, implementação de variabilidades, aplicações
Android, plataforma de computação móvel, estudos empíricos, suporte multi-versão de API

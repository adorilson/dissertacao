% Resumo em língua estrangeira (em inglês Abstract, em espanhol Resumen, em francês Résumé

\begin{center}
	\Large{\textsc{\textbf{Abstract}}}
\end{center}

\noindent Android is currently the most popular platform for the development
of mobile applications, representing more than 80\% of the operating systems market
for mobile devices. This causes demands for application customizations to handle
different devices such as screen size, processing power and available memory, languages,
and specific user needs. Twenty-three new versions of Android platform have been
released since its first release. In order to enable the successful execution of
applications on different devices, it is essential to support multiple versions of
the Application Programming Interface (API). This dissertation aims to analyze,
characterize and compare techniques used by Android applications to support multiple
versions of the API. In particular, the work seeks: (i) to identify the used techniques
to support multiple versions of the Android API in the literature; (ii) to analyze real
applications to quantify the usage of these techniques; and (iii) to compare the
characteristics and consequences of using such techniques. An empirical study, in which
25 popular Android apps were analyzed, was conducted to achieve this goal. The results
of the study show that there are three techniques to support multiple versions of the
API:
i) compatibility package, that adrresses API coarse granularity variabilities involving
a set of classes; 
ii) re-implementation of resource used for specific situations and coarse granularity
at class level or when resource is not available in compatibility package; and
iii)explicit use of the new API that allows implementing fine grained variabilities of the
API that involves
calling of specific methods. Through the analysis of 25 applications, we have identified
that compatibility package was used by 23 applications, re-implementation of resource was
used by 14 applications and the explicit usage of the new API was used by 22 applications.
The API fragments contains the most common elements among those released in higher versions
of the platform that are used by applications during their evolution, and it is referenced
by 68\% of them. In general, applications could increase their potential market with adaptations
of, on average, 15 code snippets. On the other hand, application developers have been worried
about how avoiding dead code based on platform API. In the analysis of 7 applications, 4 of
them contained dead code, but it did not represent more than 0.1\% of total code.

\noindent\textbf{Keywords}:  Variability management, implementation of variabilities,
Android applications, mobile computing platform, empirical studies, multi-version API support.

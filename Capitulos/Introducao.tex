% Introdução
\chapter{Introdução} \label{ch:introducao}

Dispositivos móveis estão ficando cada vez mais populares e acessíveis para pessoas
de diferentes poder aquisitivo ao redor do mundo \cite{Lhamas2014}. A plataforma
Android é atualmente a mais popular para o desenvolvimento de aplicações móveis,
ocupando mais de 80\% do mercado de sistemas operacionais para aplicações de
dispositivos móveis \cite{jim2014}. Tal realidade cria uma demanda por customizações
de aplicações para lidar com diferentes dispositivos, tais como tamanho de tela,
bibliotecas de classes \abrv[API -- \textit{Application Programming Interface}]
{API}(\textit{Application Programming Interface}), disponibilidade de poder de
processamento e memória disponível, idiomas e necessidades específicas dos usuários.
O Facebook, por exemplo, disponibiliza duas versões do seu aplicativo. A versão
padrão é direcionada para os dispositivos mais modernos e a alternativa, chamada
de Facebook Lite\footnote{Disponível em: https://play.google.com/store/apps/details?id=com.facebook.lite},
para dispositivos mais antigos (que utilizam a versão 2.2 do Android.), que ocupa
bem menos espaço de armazenamento, é mais eficiente no uso de dados móveis, pode
ser carregada na memória do dispositivo mais rapidamente e funciona com conexões
de Internet instáveis ou lentas. A existência de diferentes dispositivos para os
quais a plataforma Android oferece suporte é conhecido como fragmentação, e isso
se apresenta como um grande desafio para o Android, em relação a outras plataformas.
Há um número considerável de dispositivos com versões antigas da API \cite{Gronli2014}.
Diferentes versões da API é uma das variabilidades mais comuns na plataforma Android.
Esse cenário leva os desenvolvedores de aplicações Android buscarem técnicas e
metodologias que otimizem o desenvolvimento de versões da aplicação que melhor
atendam às restrições e características de cada dispositivo.

De fato, customizações em larga escala de aplicações são um requisito fundamental
que vem sendo considerado por abordagens de engenharia de linhas de produtos de
software  \cite{Alves2007} \cite{Alves2005}, direcionando esforços para a
gerência de variabilidade no contexto de aplicações móveis. Estudos \cite{Dehlinger2011}
indicam que o desenvolvimento de linhas de produtos de software para aplicações
móveis é um desafio atual da engenharia de software, enquanto outros
\cite{Linares-Vasquez2013} apontam que fazer uso de API propensas a mudanças e
falhas pode ter um impacto negativo no sucesso de aplicações. A documentação oficial
do Android \cite{GuiaAndroid} é rica em propor soluções para tais variabilidades,
no entanto, a implementação dessas soluções pode sofrer variações. Cada equipe
de desenvolvimento pode tomar decisões de projeto diferentes.

Existe atualmente uma carência de estudos específicos sobre gerência de variabilidades
em aplicações para plataforma Android. Sobretudo em relação ao suporte à múltiplas
versões da API. Neste contexto, este trabalho apresenta uma caracterização de com
o suporte a múltiplas versões da API é implementado por aplicações atuais e qual
o impacto de tais soluções no desenvolvimento de aplicações reais na plataforma
Android.

Este capítulo introduz este trabalho. A seção \ref{sec:apresentacao-do-problema}
exibe o contexto do trabalho e apresenta o problema a ser tratado. A seção \ref{sec:limitacao-abordagens-atuais}
apresenta as principais limitações das trabalhos atuais. A seção \ref{sec:objetivos-gerais-especificos}
discute os objetivos gerais e específicos, e a metodologia adotada para o desenvolvimento
do trabalho é apresentada na seção \ref{sec:metodologia}. Por fim, a organização do trabalho
é mostrada na seção \ref{sec:organizacao-trabalho}.

\section{Apresentação do Problema} \label{sec:apresentacao-do-problema}

Apesar da ampla adoção da plataforma Android e sua fragmentação, há uma carência
de estudos que indiquem de que forma as aplicações de tal plataforma lidam com
variabilidades relacionadas à diversas versões da API.

Oferecer suporte a múltiplas versões da API, de forma a atender clientes com
versões antigas ao mesmo tempo usar os recursos mais modernos quando disponíveis,
é primordial para o sucesso das aplicações.

No entanto, a carência de estudos sistemáticos que identifiquem e analisem técnicas
para implementação de variabilidades ao longo da evolução de tais aplicações e da
própria plataforma Android pode trazer dificuldades para a evolução do código de
tais aplicações ou até mesmo erros durante a execução.

\section{Limitações das Trabalhos Existentes} \label{sec:limitacao-abordagens-atuais}
Diversos trabalhos têm sido realizados sobre a API Android. Sobretudo, no sentido
de identificar os elementos mais comuns e padrões de uso. Mojica \cite{Mojica2014}
analisa como reuso de software ocorre nas aplicações disponíveis na Google Play.
Os três principais tipos de reuso identificados foram: (i) 18,75\% das classes
herdam de uma classe base da API Android, e 35,78\% de uma classe específica de
domínio; (ii) 84,23\% das classes aparecem em duas ou mais aplicações; (iii)
17.109 aplicações móveis são cópias diretas de outras aplicações. Lambda et al. \cite{Lamba2015}
apresentou resultados de um estudo em larga escala, consistindo de 1.120 aplicações
open-source e 17,4 milhões de linha de código, de análise do uso da API em aplicações
Android. Foram identificados os métodos mais frequentemente invocados, os pacotes da
API e padrões de chamadas mais populares. McDonnell et al. \cite{McDonnell2013}
realizou um estudo empírico da adoção de atualização da API pelas aplicações,
onde foi confirmado que, apesar dos benefícios de novas ou atualização de API,
desenvolvedores são frequentemente lentos na adoção das novas API’s.
 
No entanto, nenhum desses trabalhos considerou o suporte a múltiplas versões da API.
Todas as análises foram feitas sem nenhum tipo de discussão sobre a versão mínima da
API para a aplicação e a versão da API em que os elementos foram disponibilizados
pela plataforma.

\section{Objetivos Gerais e Específicos} \label{sec:objetivos-gerais-especificos}

O objetivo geral desta dissertação de mestrado é analisar, caracterizar e comparar
técnicas de implementação de suporte a múltiplas versões da API da plataforma
Android por aplicações.

Os objetivos específicos são:
\begin{itemize}
	\item Identificar na literatura quais as técnicas indicadas para suporte a
	múltiplas versões da API Android;
	\item Analisar aplicações reais para quantificar o uso dessas técnicas
	indicadas; e
	\item Comparar as características e consequências do uso de tais técnicas.
\end{itemize}

\section{Metodologia} \label{sec:metodologia}

Visando atingir os objetivos dessa dissertação de mestrado, foi definido uma
metodologia de trabalhos composta das seguintes etapas:
\begin{itemize}
	\item Revisão da literatura: nessa fase, identificamos quais as técnicas
	indicadas para suporte a múltiplas versões da API;
	\item Estudo empírico: a partir dos resultados da fase anterior, foi realizado
	um empírico com o objetivo de identificar a adoção dessas técnicas por aplicações reais.
\end{itemize}


\section{Organização do Documento} \label{sec:organizacao-trabalho}

O restante deste documento está organizado da seguinte forma:
\begin{itemize}
    \item O capítulo \ref{ch:fundamentacao-teorica} apresenta os principais
    referenciais teóricos que serviram de arcabouço para o desenvolvimento  deste trabalho;
    \item O capítulo \ref{ch:estudo-empirico} traz o estudo sobre a compatibilidade
    de aplicações Android com as diferentes versões da API da plataforma. São
    apresentados resultados do estudo e discussão de tais resultados;
    \item No capítulo \ref{ch:trabalhos-relacionados} são discutidos trabalhos
    relacionados;
    \item Por fim, o capítulo \ref{ch:conclusao} apresenta as considerações finais,
    principais contribuições e limitações do trabalho, bem como trabalhos futuros
    de pesquisa que podem ser desenvolvidos como continuidade desta dissertação.
\end{itemize}

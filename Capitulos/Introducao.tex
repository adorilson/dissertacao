% Introdução
\chapter{Introdução} \label{ch:introducao}

Dispositivos móveis estão ficando cada vez mais populares e acessíveis para
pessoas de diferentes poder aquisitivo ao redor do mundo \cite{Lhamas2014}.
A plataforma Android é atualmente a mais popular para o desenvolvimento de
aplicações móveis, ocupando mais de 80\% do mercado de sistemas operacionais
para aplicações de dispositivos móveis \cite{jim2014}. Tal realidade cria uma
demanda por customizações de aplicações para lidar com diferentes dispositivos,
tais como, tamanho de tela, bibliotecas de classes 
disponíveis \abrv[API -- \textit{Application
Programming Interface}] (API - \textit{Application Programming Interface}),
disponibilidade de poder de processamento e memória, idiomas e
necessidades específicas dos usuários. O Facebook, por exemplo, disponibiliza
duas versões do seu aplicativo. A versão padrão é direcionada para os dispositivos
mais modernos e a alternativa, chamada de Facebook Lite\footnote{Disponível em:
https://play.google.com/store/apps/details?id=com.facebook.lite},
para dispositivos mais antigos (que utilizam a partir da versão 2.2 do Android.),
que ocupa bem menos espaço de armazenamento, pode ser carregada na memória do
dispositivo mais rapidamente e funciona com conexões de Internet instáveis ou lentas.

A existência de diferentes dispositivos para os quais a plataforma Android oferece
suporte é conhecido como fragmentação, e isso se apresenta como um grande desafio
em relação à outras plataformas. Há um número considerável de dispositivos com versões
antigas da API  \cite{Gronli2014}. Já em 2011, 86\% dos desenvolvedores consideravam
a fragmentação da plataforma um sério problema \cite{Elmer-DeWitt2011}.
Diferentes versões da API é uma das variabilidades mais comuns na plataforma Android.
Esse cenário leva os desenvolvedores de aplicações Android a buscarem técnicas e
metodologias que otimizem o desenvolvimento de versões da aplicação que melhor
atendam às restrições e características de cada dispositivo. A documentação oficial
do Android \cite{GuiaAndroid} é rica em propor soluções para tais variabilidades,
no entanto, a implementação dessas soluções pode sofrer variações. Cada equipe
de desenvolvimento pode tomar decisões de projeto diferentes. Em especial, existe
atualmente uma carência de estudos sobre a forma como aplicações Android atuais
oferecem suporte às múltiplas versões da API. Neste contexto, este trabalho busca
entender as técnicas utilizadas para oferecer tal suporte e qual o impacto do seu uso.

Este capítulo introduz este trabalho. A seção \ref{sec:apresentacao-do-problema}
exibe o contexto do trabalho e apresenta o problema a ser tratado. A seção \ref{sec:limitacao-abordagens-atuais}
apresenta as principais limitações das trabalhos atuais. A seção \ref{sec:objetivos-gerais-especificos}
discute os objetivos gerais e específicos, e a metodologia adotada para o desenvolvimento
do trabalho é apresentada na seção \ref{sec:metodologia}. Por fim, a organização do trabalho
é mostrada na seção \ref{sec:organizacao-trabalho}.

\section{Apresentação do Problema} \label{sec:apresentacao-do-problema}

Apesar da ampla adoção da plataforma Android e o problema da fragmentação \cite{Park2013},
há uma carência de estudos que indiquem de que forma as aplicações de tal plataforma
lidam e se adaptam às diversas versões lançadas da API. 
Enquanto atender a usuários com uma antiga versão da API significa ter um amplo
mercado potencial, utilizar os recursos das versões mais novas da API é um dos três 
fatores mais importantes para as aplicações com alta avaliação na Google Play \cite{Tian2015}.
Oferecer suporte a múltiplas versões da API da plataforma, de forma a atender clientes
com versões antigas ao mesmo tempo usar os recursos mais modernos quando disponíveis,
é primordial para o sucesso das aplicações.

Uma alternativa possível é disponibilizar uma versão da aplicação para cada versão
da API. Essa solução é a comumente utilizada para API’s locais - API’s de sistemas
operacionais de computadores de mesa, por exemplo -, onde o número de versões de
API’s é reduzido e a evolução é mais lenta em comparação à API do Android: existem
cerca de 10 versões da API com uma margem relevante do mercado. Manter diversas
versões simultâneas de uma aplicação causa duplicação de código e prejudica a
manutenabilidade. % TODO colocar referencia aqui
Assim, a plataforma oferece mecanismos para que uma única base
de código atenda da melhor forma possível às diversas versões da API, usando os
melhores recursos disponíveis nos aparelhos.

No entanto, no melhor do nosso conhecimento, não existem estudos sistemáticos que
identifiquem e analisem técnicas para implementação de suporte às
múltiplas versões da plataforma Android. Tais estudos podem trazer luz sobre as
práticas adotadas pela comunidade de desenvolvedores e auxiliá-los a decidir
quando e quais técnicas melhor atendem a determinadas situações.

\section{Limitações das Trabalhos Existentes} \label{sec:limitacao-abordagens-atuais}
Diversos trabalhos têm sido realizados sobre a API Android. Sobretudo, no sentido
de identificar os elementos mais comuns e padrões de uso. Lamba et al.\cite{Lamba2015} apresentaram
resultados de um estudo em larga escala de análise do uso da API em aplicações Android.
O estudo envolveu 1.120 aplicações open-source e 17,4 milhões de linha de código.
Foram identificados os métodos mais frequentemente invocados, os pacotes da API,
classes, e padrões de chamadas mais populares.Os métodos mais comuns são \texttt{getString},
\texttt{get} e \texttt{toString}. Os pacotes mais comuns são \texttt{java.util} e
\texttt{android.content}. As classes mais comuns são \texttt{Context} e \texttt{View}.
O padrão de chamada mais popular está relacionado à dimensão, espaçamento e margens
de uma \textit{view} na interface com o usuário. No entanto, na análise de popularidade
de métodos não levaram em consideração suas respectivas classes. Por exemplo, o método
\texttt{getString} da classe \texttt{Activity} foi contabilizado juntamente com
o método \texttt{getString} da classe \texttt{Cursor}, que não possuem nenhuma
relação entre si.

McDonnell et al. \cite{McDonnell2013} conduziram um estudo sobre a co-evolução
da API Android e suas aplicações usando o histórico de versões encontradas no
Github. O estudo confirmou que a plataforma Android evolui rapidamente em uma
taxa de 115 mudanças da API por mês na média. Contudo, a adoção pelas aplicações
clientes não segue na mesma velocidade. Cerca de 28\% das referências à API nas
aplicações clientes estão desatualizadas com uma mediana de tempo de 16 meses.
22\% das referências desatualizadas são eventualmente atualizadas para versões
mais novas da API, mas com um tempo de propagação de cerca de 14 meses, que é mais
lento que o tempo médio entre novas versões da API (3 meses). APIs de rápida
evolução são mais usadas por clientes que APIs de evolução lenta, mas o tempo
médio necessário para adoção de novas versões é mais longo para APIs de rápida
evolução. Além disso, código adaptado para uso de novas APIs são mais sujeitos a
erros que aqueles sem adaptação para novas APIs. Segundo os autores do trabalho,
os resultados sugerem que os desenvolvedores não adotam novas API’s com rapidez,
mantendo referências para API’s desatualizadas. Assim evitam a instabilidade de
novas API’s e o trabalho de atualização propriamente dito. Porém, outras possibilidades
devem ser consideradas: os desenvolvedores atualizam para as mais recentes API’s,
mas mantém tais códigos para garantir a compatibilidade com API’s mais antigas;
ou atualizam e as referências às API’s antigas são código-morto.

Aplicações Android declaram uma versão alvo da API. Quando estão em execução em
um dispositivo cuja versão da API é inferior à versão alvo, as aplicações são
executadas em um modo de compatibilidade, que busca reproduzir o comportamento
da versão alvo indicada pelo aplicação. Esse cenário desabilita as melhorias da
nova versão, incluindo as correções de problemas de segurança. Mutchler \cite{Mutchler2016}
chamou essa característica de problema da fragmentação do alvo e analisou um
conjunto de 1.232.696 aplicações Android para mostrar que isso traz sérias
consequências em todo ecossistema de aplicações e não mudou consideravelmente ao
passar dos anos. No total, 93\% das aplicações atualmente definem como alvo uma
versão desatualizada e possuem uma média de desatualização de 686 dias; 79\% das
aplicações já estão desatualizadas no dia que são carregadas para o Google Play.
Também analisaram 5 mudanças na plataforma relacionadas a segurança e que são
desabilitadas nessas aplicações desatualizadas.

Esses trabalhos não abordam o tema de suporte a múltiplas versões da API por aplicações
Android existentes. Todas as análises de estudos existentes foram realizadas sem
abordar questões sobre a adaptação e organização do código para lidar com as
múltiplas APIs disponíveis.

Wei \cite{Wei2016} realizou um estudo sobre problemas de compatibilidade induzidos
pela fragmentação 
\abrv[FIC -- \textit{Fragmentation-Induced Compatibility}]
(FIC issues - \textit{Fragmentation-Induced Compatibility issues}).
De forma manual, analisaram 191 \textit{issues} nos sistemas de reporte de erros
de 5 projetos de código aberto e os \textit{commits} relacionadas a esses \textit{issues}.
As questões de pesquisa desse trabalho foram:
\begin{itemize}
	\item \textbf{QP1}: Quais são os tipos comuns de FIC issues em aplicações
	Android? Quais são suas causas?
	    \begin{itemize}
	        \item \textbf{R}: Observaram que existem 5 maiores causas de FIC \textit{issues}
	        em aplicações Android, entre elas a evolução da API da plataforma e implementação
	        problemática de drive de hardware. Todos os tipos e causas são apresentados
	        na Tabela \ref{tab:causas_FIC};
	        \begin{table}[h] %  TODO centralizar tabela na pagina
              \centering
              \caption{Tipos e causas de FIC issues \cite{Wei2016}}
              \begin{tabular}{ | l | l |}
                \hline % TODO a primeira coluna deve ser multiplas linhas
                \textbf{Tipo} & \textbf{Causa}  \\ \hline
                \multirow{3}{*}{Específico do dispositivo} & Implementação de hardware problemática   \\ \cline{2-2}
                                                         & Customização do SO \\ \cline{2-2}
                                                         & Composição de hardware peculiar \\ \hline
                \multirow{2}{*}{Não-específico do dispositivo} & Evolução da API da plataforma Android \\ \cline{2-2}
                                                             & Bugs do sistema Android Original \\ \hline
              \end{tabular}
              \label{tab:causas_FIC}
            \end{table}
	    \end{itemize}
	\item \textbf{QP2}: Quais são os sintomas mais comuns de FIC \textit{issues} em aplicações
	Android?
	    \begin{itemize}
	        \item \textbf{R}: FIC \textit{issues} podem causar tanto consequências funcionais
	        como não-funcionais, como travamento da aplicação, aplicação não funcionar,
	        degradação na performance e experiência de uso. Os sintomas podem ser
	        específicos da aplicação, o que dificulta a criação de testes de
	        compatibilidade;
	    \end{itemize}
	\item \textbf{QP3}: Como os desenvolvedores corrigem FIC \textit{issues} na
	prática?
	Existe algum padrão comum?
	    \begin{itemize}
	    \item \textbf{R}: Localizar a causa dos FIC \textit{issues} é difícil na
	    prática.
	    Por outro lado, as correções são usualmente simples e apresentam padrões
	    comuns: checar informações do dispositivo e disponibilidade de componentes
	    de software/hardware antes de chamar APIs/métodos causadores de problemas.
	    Este é o padrão mais comum, presente em 137 das 191 correções analisadas.
	    Uma forma típica de evitar os erros é pular a chamada da API ou substituí-la
	    por uma implementação alternativa.
	    \end{itemize}
\end{itemize}

Também desenvolveram uma ferramenta, FicFinder, que realiza uma análise estática
nas aplicações e informa se estas contém FIC \textit{issues}. Realizaram um experimento
com 27 aplicações para avaliar a ferramenta a responder às seguintes questões de
pesquisa:
\begin{itemize}
    \item \textbf{QP4}: O FicFinder pode ajudar a detectar FIC \textit{issues} em
    aplicações Android do mundo real?
        \begin{itemize}
            \item \textbf{R}:FicFinder efetivamente detectou FIC \textit{issues}
            desconhecidos em aplicações Android do mundo real e obteve uma alta
            precisão na análise. Nas 27 aplicações foram emitidos 51 alertas,
            sendo 46 de casos confirmados (precisão de 90,2\%);
        \end{itemize}
    \item \textbf{QP5}: FicFinder fornece informações úteis aos desenvolvedores
    de aplicações para facilitar o diagnóstico e processo de correção de FIC
    \textit{issues}?
        \begin{itemize}
            \item \textbf{R}: FicFinder pode fornecer informações úteis para ajudar
            desenvolvedores a diagnosticar e corrigir FIC \textit{issues}. Os
            alertas emitidos pelo FicFinder foram encaminhados para os desenvolvedores
            das aplicações. Alguns já haviam corrigido o problema e outros informaram
            que iriam corrigir.
        \end{itemize}
\end{itemize}

No experimento, o FicFinder foi alterado para reportar situações em que os FIC
\textit{issues} já estavam protegidos de execuções indevidas, tais casos foram
identificados como “boas práticas”. Porém, o trabalho não apresentou essas boas
práticas, tampouco fez qualquer tipo de comparação, avaliação ou indicação de uso.

\section{Objetivos} \label{sec:objetivos-gerais-especificos}

O objetivo geral desta dissertação de mestrado é analisar, caracterizar e comparar
técnicas de implementação de suporte a múltiplas versões da API da plataforma
Android por aplicações existentes. Os objetivos específicos do trabalho são:
\begin{itemize}
	\item Identificar na literatura quais as técnicas indicadas para suporte a
	múltiplas versões da API Android;
	\item Analisar aplicações reais para quantificar o uso dessas técnicas indicadas;
	\item Comparar as características e consequências do uso de tais técnicas; e
	\item Identificar as mudanças mais comuns na API que afeta a evolução das
	aplicações para rodar em aparelhos com versões diferentes da API.
\end{itemize}

\section{Metodologia} \label{sec:metodologia}

Visando atingir os objetivos dessa dissertação de mestrado, foi definido uma
metodologia de trabalho composta das seguintes etapas:
\begin{itemize}
	\item Revisão da literatura: essa etapa buscou identificar quais as técnicas
	indicadas pela literatura para oferecer suporte a múltiplas versões da API,
	além de entender melhor os trabalhos relacionados atuais;
	\item Estudo empírico: a partir dos resultados da fase anterior, foi realizado
	um estudo empírico com o objetivo de identificar e caracterizar a adoção dessas
	técnicas por aplicações reais. Foi composto das seguintes atividades:
	\begin{itemize}
	    \item Seleção das aplicações;
	    \item Análise das aplicações, e;
	    \item Análise dos resultados.    
	\end{itemize}
\end{itemize}


\section{Organização do Documento} \label{sec:organizacao-trabalho}

O restante deste documento está organizado da seguinte forma:
\begin{itemize}
    \item O capítulo \ref{ch:fundamentacao-teorica} os principais referenciais
    teóricos que serviram de arcabouço para o desenvolvimento deste trabalho;
    \item O capítulo \ref{ch:estudo-empirico}  traz o estudo sobre a compatibilidade
    de aplicações Android com as diferentes versões da API da plataforma. São
    apresentados resultados do estudo e discussão de tais resultados;
    \item No capítulo \ref{ch:trabalhos-relacionados} são discutidos trabalhos
    relacionados;
    \item Por fim, o capítulo \ref{ch:conclusao} apresenta as considerações finais,
    principais contribuições e limitações do trabalho, bem como trabalhos futuros
    de pesquisa que podem ser desenvolvidos como continuidade desta dissertação.
\end{itemize}

% Capíulo 4
\chapter{Trabalhos Relacionados} \label{ch:trabalhos-relacionados}

Trabalhos de pesquisa relacionados à evolução de API e suporte a múltiplas versões
de API, sejam da plataforma Android ou outras tecnologias, e popularidade dos
elementos da API Android são relevantes para este trabalho. Em especial, verificamos
que não existem muitos trabalhos relacionados de suporte à múltiplas versões da API
do Android. Assim, a principal contribuição deste trabalho é de caracterização das
técnicas de implementação de suporte a múltiplas versões da API Android em aplicações
reais. As seções seguintes detalham trabalhos de pesquisa relacionados, organizados
por tópicos no qual oferecem contribuição relevante.

\section{Evolução da API Android}

McDonnell et al. \cite{McDonnell2013} conduziram um estudo sobre a co-evolução da API
Android e suas aplicações usando o histórico de versões encontradas no Github, comparando
a velocidade de evolução da API Android com a adoção pelas aplicações. No entanto, não
considerou o suporte multi-versão da API. O nosso estudo foi focado nesse aspecto. Embora
as aplicações permaneçam oferecendo suporte para API antigas, elas buscam usar recursos das
mais novas quando estão disponíveis ou através do pacote de compatibilidade ou re-implementação
de recurso.

Diferentemente deste trabalho, que considerou a API como um todo, alguns trabalhos focaram
em componentes específicos, como Wei et al.\cite{Wei2012} que abordaram a evolução do sistema
de permissões para acessar recursos de \textit{hardware} e \textit{software} da plataforma.
No início de 2009, a API 3 possuía 103 permissões, enquanto na versão 15 esse número já chegava
à 165. As aplicações também precisam adaptar-se à essa evolução. 

Neste trabalho, não abordamos elementos de API que foram depreciados, mas alguns trabalhos
focam seus estudos nesta perspectiva.  Zhou e Walker \cite{Zhou2016} propuseram a ferramenta
\textit{Deprecation Watcher} para identificar uso de API depreciadas em código-fonte de exemplos
na web, assim desenvolvedores podem ser informados antes de investir tempo e esforço estudando-os.
A ferramenta foi avaliada na detecção de APIs Android depreciadas utilizadas em código-fonte de exemplo no Stack Overflow. Eles obtiveram uma precisão de 100\% e uma cobertura de 86\% em um
conjunto aleatório de 200 questões.

\section{Fragmentação da API Android}

Mutchler et al. \cite{Mutchler2016} abordaram o problema da fragmentação da API alvo.
Esse problema ocorre quando as aplicações não atualizam sua versão de API alvo para a
mais recente da plataforma. Essa falta de atualização desabilita melhorias da plataforma
mesmo quando as aplicações são executadas em dispositivos com a versão mais recente da API.
Um exemplo é a vulnerabilidade que permite injeção de fragmentos. Aplicações maliciosas
podem enviar mensagens para classes que herdam de
\texttt{Preferen ceActivity}, % esse espaço é gambiarra pra quebra de linha
existente
desde a primeira versão do Android. As mensagens são interpretadas como instâncias de
fragmentos e carregadas dinamicamente, executando código arbitrário da aplicação
mal-intencionada. A API nível 19 adicionou o método \texttt{isValidFragment} para resolver
esse problema. Aplicativos que têm como alvo os níveis de API 19 ou superiores herdam uma implementação padrão de \texttt{isValidFragment} que sempre lança uma exceção e, portanto,
são seguros por padrão, mas aqueles que têm como alvo os níveis 18 ou inferior herdam uma implementação que sempre retorna \texttt{true}, não oferecendo proteção contra esse ataque,
ainda que em execução em um dispositivo com versão superior a 18.  Como API 19 é uma API
alta, provavelmente a versão mínima das aplicações  será bem inferior, exigindo que seja
utilizado um execução condicional na chamada à \texttt{isValidFragment}. O nosso trabalho
complementa este apresentando soluções para oferecer suporte múltiplas versões da plataforma
e indicando os usos mais comuns. Assim, as aplicações podem usar os melhores recursos de cada
versão da plataforma. 

Wei et al. \cite{Wei2016} identificaram tipos e causas para o problema de compatibilidade
induzidos pela fragmentação (FIC issues - Fragmentation-Induced Compatibility issues), um
dos tipos é não-específico de dispositivos, que pode ser causado devido a evolução da API da
plataforma, impactando no suporte a múltiplas versões da API. Também apresentaram uma ferramenta, 
FicFinder, para identificar tais problemas nas aplicações. O nosso trabalho utilizou uma ferramenta 
nativa do ambiente de desenvolvimento oficial, o Android Lint (regra NewApi). Assim como o nosso 
trabalho, eles também identificaram os usos de novas APIs em métodos executados pela plataforma. Nós 
chamamos de suporte implícito (Seção \ref{subsec:tecnicas_usadas}), enquanto eles reportaram como 
falsos positivos. Na verdade, tanto o FicFinder quanto o Android Lint poderiam ter um catálogo desses 
métodos e não reportarem esses resultados para os desenvolvedores. Além disso, quantificamos as boas 
práticas citadas por eles para corrigir ou prevenir os FIC \textit{issues} causadas pela evolução da API da plataforma.

\section{Popularidade dos elementos da API Android}

O estudo feito por Lamba et al. \cite{Lamba2015} considerou a popularidade dos
elementos da API de forma absoluta, sem levar em consideração a versão da API
exigida pela aplicação e a versão da API em que o elemento foi disponibilizado.
Além disso, a quantificação foi feita por níveis diferentes de granularidade, do
mais fino para o mais grosso: método, classe e pacote. Nosso trabalho considerou
apenas os elementos que surgiram após a versão da aplicação e os resultados foram
agrupados, primariamente, por uma taxonomia, e só após apresentamos métodos, classes
e pacotes das taxonomias mais comuns. Os métodos mais comuns encontrados por eles são
\texttt{getString()}, \texttt{get()} e \texttt{toString()}. O nosso estudo apresentou
métodos da classe \texttt{Activity} relacionadas a API de fragmentos e o método
\texttt{bigContentView()} da classe \texttt{Notification} como os mais comuns
(contabilizado por uso explícito de nova API - Seção \ref{subsec:subjunto_api}). No
nível de classe, eles apontaram \texttt{Context} e \texttt{View} como as mais comuns.
Enquanto que encontramos \texttt{Fragment}, \texttt{NotificationCompat} e \texttt{RecyclerView}
(um subtipo de \texttt{View}) como as mais comuns (contabilizado por número de
\texttt{import}’s do pacote de compatibilidade - Seção \ref{subsec:subjunto_api}).
Eles apontaram os pacotes \texttt{java.util} e \texttt{android.content} como os mais comuns.
Não fizemos muitas sumarizações por pacotes, apenas o pacote \texttt{android.app.backup}
apareceu durante a análise por uso explícito da nova API de aplicações que exigiam API nível
4. O pacote surgiu na API nível 5. Por outro lado, os pacotes \texttt{java.util} e
\texttt{android.content} correspondem às nossas taxonomias \texttt{java.util} e \texttt{content}.
A \texttt{java.util} apareceu uma única vez em nossa análise. Os resultados combinados nos
levam a conclusão que esse pacote é bastante utilizado e muito estável. Já a taxonomia
\texttt{content} encontra-se em um posição intermediária em nossos resultados.

\section{Compatibilidade ou suporte a múltiplas versões de API}

A plataforma Android permite compatibilidade simultânea com múltiplas versões
da API e uma política conservadora em relação à depreciação, de forma que os
métodos e classes permanecem nas novas versões mesmo de depois anotados como
depreciados. Alguns trabalhos buscaram entender a política de depreciação e
suporte a múltiplas versões em outras plataformas. Espinha et al. \cite{Espinha2014}
analisaram a política de evolução de quatro APIs web: Google Maps, Twitter, 
Netflix e Facebook. Os três primeiros possuem um sistema de versionamento
explícito, em que um identificador da versão é incluída na URL de acesso à API
(ex.: \texttt{domain/<version>/method}), o que permite aos clientes escolherem qual
versão da API irão utilizar. Essa abordagem facilita a migração dos clientes, bastando
alterar o trecho da URL referente à versão  Porém, uma evolução de API pode ir além que
somente adição ou remoção de métodos. Por exemplo, a evolução da Twitter API analisada
no trabalho trata da versão 1 para 1.0. Além da mudança na URL (por exemplo, de
\texttt{twitter.com/1/method} para \texttt{twitter.com/1.1/method}) outras elementos
também foram modificados. Por exemplo, os clientes foram forçados a autenticar,
o suporte para XML foi descontinuado em favor de JSON e mudanças foram feitas para limitar
o acesso (que pode penalizar clientes com alta frequência de acesso). Para esses casos,
é necessário uma intervenção mais profunda no código das aplicações clientes. O estudo
foi complementado com um conjunto de recomendações para os provedores de API para
facilitar a evolução dos clientes, entre elas: i) não mudar tão frequentemente, para
que os cliente não se vejam também obrigados a alterar suas aplicações com frequência;
e ii) não deixar as versões antigas por muito tempo, pois quando  isso ocorre os
desenvolvedores das aplicações clientes relaxam e atrasam muito a atualização.
Ficando muito distante da versão mais recente, a atualização pode tornar-se difícil.
A disponibilidade da versão da API no Android, através de \texttt{Build.VERSION},
permitindo execução condicional, é um mecanismo para versionamento. Já os pacotes
de compatibilidade e a re-implementação de recursos são alternativas a ele.

Sohan et al. \cite{Sohan2015} realizaram um estudo de caso de evolução da API
web de 9 serviços e identificaram dois modos de atualização: versão única e
múltiplas versões. No modo de versão única, as APIs web forçam a migração dos
clientes para a nova versão da API removendo a versão anterior após um período
de tempo, que pode ser conhecido, como nos casos da Facebook REST API (90 dias)
e Google Calendar (6 meses), ou desconhecido, como nos casos da Twitter REST API,
Github API OpenStreetMap API e Google Maps. O modo de múltiplas mantém as APIs
antigas por um longo período de tempo. Wordpress REST API, Salesforce API e Stripe
API essa estratégia. Tal estratégia permite uma maior flexibilidade para os clientes,
mas exige maior esforço dos fornecedores da API, como resultado nem todas as APIs são
mantidas. Quando os clientes atrasam muito a atualizando, ficando muito distante da
versão mais recente, a atualização pode tornar-se difícil. O mesmo pode acontecer
com aplicações Android. Foram também analisados diferentes abordagens para identificar
a versão da API: i) identificador numérico: números inteiros contínuos que comunicam
claramente uma sequência entre as versões; ii) identificador com data: datas também
pode ser usadas para identificar versões; e iii) identificadores com números de versões
principal e secundário (ex: 2.3): a maioria das APIs web analisadas utilizam essa
abordagem. Nenhuma dessas abordagem, por si só, garantem compatibilidade entre as versões.
Inclusive, frequentemente as APIs são atualizadas com quebra de compatibilidade sem receber
um novo identificador de versão. A abordagem de identificador número é a adotada pela
plataforma Android, sendo a compatibilidade garantida por contrato \cite{ApiLevels},
ou seja, as novas versões são projetadas de forma a manter a compatibilidade com suas
versões anteriores.   Importante destacar que o acesso à APIs web é remoto, o que permite
seus clientes escolherem qual versão utilizarão, enquanto o acesso à API Android é local,
exigindo outras estratégias para acesso à API. Nosso trabalho identificou, e quantificou,
pacote de compatibilidade, execução condicional e re-implementação de recurso como estratégias
de suporte a múltiplas versões da API Android.
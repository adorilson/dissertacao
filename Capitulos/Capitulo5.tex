% Capíulo 5
\chapter{Conclusão e Trabalhos Futuros} \label{ch:conclusao}

Este capítulo  apresenta  as  conclusões  desta  dissertação de  mestrado.
A Seção \ref{sec:analise} apresenta  os  resultados  alcançados  para  cada 
questão  de  pesquisa. A Seção  \ref{sec:limitacoes} discute as  principais 
limitações  do  trabalho. A Seção \ref{sec:contribuicoes} ressalta suas contribuições. Finalmente, a Seção \ref{sec:trabalhos_futuros} apresenta proposições de
trabalhos futuros que estendam e avancem os resultados desta pesquisa.

\section{Análise dos Resultados da Dissertação}\label{sec:analise}

O primeiro objetivo específico era identificar na literatura quais as
técnicas indicadas para suporte a múltiplas versões da API Android. Tal
objetivo foi contemplado ao identificarmos que são três as técnicas
primárias indicadas, sobretudo pela documentação oficial para desenvolvedores
de aplicações Android: (i) pacote de compatibilidade; (ii) re-implementação de
recurso, e (iii) uso explícito da nova API.

O uso dos pacotes de compatibilidade e re-implementação de recurso possuem formas
mais uniformes de adoção. Para o primeiro, basta adicionar a dependência na aplicação,
fazer a importação dos arquivos necessários e usar as classes conforme documentação.
Para o segundo, é realizado um \textit{"copy-and-paste"} do código-fonte do recurso
desejado. Já para fazer uso explícito da nova API é necessário garantir que a chamada
à nova API somente será executada em dispositivos em que a nova API esteja presente,
sob pena de travamentos da aplicação em dispositivos antigos. Tal proteção é feita com
execução condicional. No entanto, a forma dessa execução condicional pode variar bastante. Desde uma simples execução condicional direta, o que pode acarretar em espalhamento do
código, até o uso de padrões de projeto, que permite uma maior modularização.

Pacote de compatibilidade é utilizado quando se deseja um conjunto de novos recursos,
cujo implementação está distribuída em diversos arquivos. No entanto, deve-se analisar
a relação entre a quantidade de recursos utilizados e o aumento no tamanho do instalador
da aplicação. Se for desfavorável, pode-se fazer re-implementação de recurso do item específico. Re-implementação de recurso também pode ser utilizado quando o recurso desejado
não está disponível por meio de pacote de compatibilidade. Porém, os desenvolvedores ficarão 
responsável por manter mais um elemento, juntamente com o restante da aplicação. Essas duas 
técnicas permitem o uso dos recursos em todos os dispositivos que a aplicação for instalada, 
independente da versão da API, e são utilizadas para uso de recursos em granularidade grossa,
a partir de classes. Para os casos de granularidade fina, chamadas a métodos de classes 
pré-existentes, e quando o recurso pode deixar de ser utilizado em dispositivos com versões 
antigas da API sem prejuízos para o funcionamento da aplicação são feitos acessos diretos a 
nova API.

No nosso estudo analisamos 25 aplicações para quantificar o uso desses mecanismos.
Pacote de compatibilidade é utilizado por 92\% (23) das aplicações, re-implementação
de recurso por 56\% (14) e 88\% (22) fazem uso explícito da nova API. 88\% (22) das
aplicações utilizam mais de um mecanismo simultaneamente. 

Nas 22 aplicações que fizeram uso explícito da nova API, totalizaram 1556 ocorrências
de NewApi. Desse total, 40,6\% (632 ocorrências) foi tratado com EC indireta; 22,8\%
(355) com algum padrão de projeto; 18,9\% (294) através de EC direta; 10,7\% (166)
utilizando suporte implícito da plataforma; 5,3\% (82) das ocorrências não foram devidamente 
tratadas; 1,2\% (19) das ocorrências estão dentro de métodos que nunca serão executados;
e por fim, duas pequenas frações de 0,3\% (5) das ocorrências foi utilizada tratamento
de exceção ou são falsos positivos. Os padrões de projeto utilizados para tratar as 355
ocorrências foram \textit{Strategy}, com 77,7\% (276) desse total, \textit{Proxy}, 
com 19,4\% (69), \textit{Template Method}, com 2,3\% (8), e \textit{Null Object},
com 0,6\% (2) das ocorrências. 

Também identificamos quais os elementos mais comuns lançados em versões superiores
à versão das aplicações mas que são utilizados por elas. Os elementos mais comuns
estão relacionados à classe \texttt{Fragment}, que surgiu no Android 3.0 API Level
11, quando a plataforma foi otimizada para dispositivos de telas grandes, como os
\textit{tablets}. Essa classe foi referenciada por 68\% (17) das aplicações, sendo
15 por pacote de compatibilidade e 2 por execução condicional.

Se por um lado detectamos que, no geral, as aplicações poderiam aumentar o seu mercado
em potencial com adaptações de, em média, 15 novas ocorrências de NewApi,  por outro
lado, identificamos que os desenvolvedores têm se preocupado em evitar código-morto
em função da API da plataforma. Na análise de 7 aplicações, 4 delas contém código-morto.
No entanto, em termos de linha de código, não passa de 0,1\% das linhas de código Java.

\section{Limitações do Trabalho}\label{sec:limitacoes}

Os resultados deste estudo forneceram evidências importantes para discutir e responder
às questões de pesquisa desta dissertação. Contudo, existem limitações na pesquisa
realizada que precisam ser destacadas para serem exploradas em trabalhos futuros.
São elas:

\begin{itemize}
	\item Foi realizada uma revisão do estado da arte para investigar a existência de
		outros trabalhos que lidem com suporte a múltiplas versões de APIs, seja restrita
		a Android ou não, mas é importante a condução de uma revisão sistemática da
		literatura que busque ampliar o escopo dos trabalhos identificados nesta dissertação;
	\item A análise foi feita de forma semi-automática. Em especial, a categorização
		das formas de implementação do acesso explícito às novas APIs foi manual, o que
		pode trazer inconsistências nos resultados;
	\item A quantificação dos elementos mais comuns foi feita baseada na importação dos
		pacotes ou a partir dos resultados do Android Lint. Assim, não foi considerado
		heranças em classes bases que seriam estendidas por outra ou encapsulamentos de
		chamadas em classes das aplicações, o que reduz o número de referência diretas aos 
		elementos.  Portanto, os elementos podem ser mais utilizados do que os resultados 
		apresentados neste trabalho;
	\item Analisamos apenas 41 aplicações únicas. O que é um número muito baixo em relação
		ao universo de aplicações Android existentes.
\end{itemize}

\section{Principais Contribuições}\label{sec:contribuicoes}

A seguir destacamos as principais contribuições desta dissertação:
\begin{itemize}
	\item Levantamento e identificação de técnicas para suporte a múltiplas versões
		da API Android;
	\item Condução de um estudo empírico para analisar como as aplicações atuais
		adotam e implementam tais técnicas;
	\item Identificação de quais elementos de novas API são mais utilizados por
		aplicações que também oferecem suporte para os aparelhos com versões antigas
		da API;
	\item Identificação das situações mais comuns em que as técnicas de suporte
		foram utilizadas;
	\item Embora inicialmente citada por McDonnell et al. \cite{McDonnell2013},
		a taxonomia da API apresentava algumas inconsistências em sua definição.
		Neste trabalho, consolidamos tais conceitos, que podem vir a ser utilizados
		por trabalhos futuros.
\end{itemize}

\section{Trabalhos Futuros}\label{sec:trabalhos_futuros}

Como trabalhos futuros, planejamos estender os resultados desta pesquisa
de modo a lidar com limitações existentes no trabalho, assim como ampliar 
nossas contribuições. Os seguintes trabalhos de pesquisa podem ser desenvolvidos
como desdobramento desta dissertação:

\begin{itemize}
	\item Realizar uma revisão sistemática da literatura para ampliar nosso
		escopo de trabalho sobre suporte a múltiplas versões de API, restritos
		ou não ao Android. Com isso, podemos realizar análises comparativas com
		outros trabalhos e também refinar nossa metodologia de análise;
	\item Ampliar a automação do processo de análise das aplicações. Com isso,
		poderemos analisar mais aplicações de uma forma mais precisa;
	\item Ampliar o número de aplicações analisadas. Tal trabalho pode ser facilitado
		pela automação do processo, além disso, daremos mais confiabilidade para os
		dados obtidos;
	\item Análise do histórico do repositório de código-fonte das aplicações para
		responder a outras questões, como "quando os padrões de projetos foram
		implementados? O uso explícito às novas APIs foram protegidos sempre com
		padrões de projeto ou inicialmente foi utilizada execução condicional simples?"; 
	\item Estudo qualitativo com desenvolvedores de aplicações Android com o objetivo
		de responder, entre outras questões, os motivos pelos quais escolheram uma ou
		outra técnica.
	\item Análise comparativa entre FicFinder e Android Lint (regra NewApi) e,
		possivelmente, outros analisadores sintáticos.
\end{itemize}

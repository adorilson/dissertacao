\chapter{Introdução}

Dispositivos móveis estão ficando cada vez mais populares e acessíveis para
pessoas de diferentes poder aquisitivo ao redor do mundo \cite{Lhamas2014}.
A plataforma Android é atualmente a mais popular para o desenvolvimento de aplicações
móveis, ocupando mais de 80\% do mercado de sistemas operacionais para aplicações
de dispositivos móveis \cite{jim2014}. Tal realidade cria uma demanda por customizações
de aplicações para lidar com diferentes dispositivos, tais como tamanho de tela,
bibliotecas de classes (API - \textit{Application Programming Interface}),
disponibilidade de poder de processamento e memória
disponível, idiomas e necessidades específicas dos usuários. O Facebook, por exemplo,
disponibiliza duas versões do seu aplicativo. A versão padrão é direcionada para
os dispositivos mais modernos e a alternativa, chamada de
Facebook Lite\footnote{Disponível em: https://play.google.com/store/apps/details?id=com.facebook.lite}
, para dispositivos mais antigos (que utilizam a versão 2.2 do Android), que
ocupa bem menos espaço de armazenamento, é mais eficiente no uso de dados móveis,
pode ser carregada na memória do dispositivo mais rapidamente e funciona com conexões
de Internet não estáveis ou lentas. A existência de diferentes dispositivos para
os quais a plataforma Android oferece suporte é conhecido como fragmentação, e isso
se apresenta como um grande desafio para o Android, em relação a outras plataformas.
Há uma número considerável de dispositivos com versões antigas da API \cite{Gronli2014}.

De fato, customizações em larga escala de aplicações são um requisito fundamental
que vem sendo considerado por abordagens de engenharia de linhas de produtos de software
\cite{Alves2007} \cite{Alves2005}, direcionando esforços para a gerência de variabilidade
no contexto de aplicações móveis. Estudos \cite{Dehlinger2011} indicam que o desenvolvimento
de linhas de produtos de software para aplicações móveis é um desafio atual da
engenharia de software, enquanto outros enquanto outros \cite{Linares-Vasquez2013}
apontam que fazer uso de API propensas a mudanças e falhas pode ter um impacto
negativo no sucesso de aplicações. A documentação oficial do Android \cite{GuiaAndroid}
é rica em propor soluções para tais variabilidades, no entanto, a implementação
dessas soluções pode sofrer variações. Cada equipe de desenvolvimento pode tomar
decisões de projeto diferentes. Existe atualmente uma carência de estudos
específicos sobre gerência de variabilidades em aplicações para plataforma Android.

Neste contexto, este trabalho apresenta uma caracterização de como tais problemas
relacionados a gerência de variabilidade são tratadas por aplicações atuais e
qual o impacto de tais soluções no desenvolvimento de aplições reais na plataforma
Android.

\section{Problema}

Apesar da ampla adoção da plataforma Android e sua fragmentação, há uma carência
de estudos que indiquem de que forma as aplicações de tal plataforma lidam com
variabilidades relacionadas a customização da mesma para diferentes dispositivos. 

\cite{FalvoJunior2015} conduziu uma revisão sistemática da literatura tendo como
principal objetivo identificar estudos que integrassem os conceitos de linha de
produtos de \textit{software} (LPS) e dispositivos móveis. Após a seleção final
dos trabalhos, os autores chegaram em um conjunto de 31 artigos que foram lidos
na íntegra. Desses, 19 foram publicados em anos anteriores a 2009, portanto, antes
da criação da plataforma Android. Dos demais trabalhos, apenas um é relacionado a
plataforma Android, o qual relata uma experiência prática da implementação de uma
linha de produto de \textit{software} para monitoramento remoto de pessoas com
algum tipo de cuidado especial. Porém, tal trabalho aborda uma proposta do uso de
diagramas UML (\textit{Unified Modeling Language}) para representar variações arquiteturais.

\section{Objetivos da Pesquisa}

\subsection{Objetivos Gerais}

\subsection{Objetivos Específicos}

\section{Justificativas}

Texto

\section{Organização do Documento}

Texto

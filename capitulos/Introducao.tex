\chapter{Introdução}

Dispositivos móveis estão ficando cada vez mais populares e acessíveis para
pessoas de diferentes poder aquisitivo ao redor do mundo \cite{Lhamas2014}.
A plataforma Android é atualmente a mais popular para o desenvolvimento de aplicações
móveis, ocupando mais de 80\% do mercado de sistemas operacionais para aplicações
de dispositivos móveis \cite{jim2014}. Tal realidade cria uma demanda por customizações
de aplicações para lidar com diferentes dispositivos, tais como tamanho de tela,
bibliotecas de classes (API - \textit{Application Programming Interface}),
disponibilidade de poder de processamento e memória
disponível, idiomas e necessidades específicas dos usuários. O Facebook, por exemplo,
disponibiliza duas versões do seu aplicativo. A versão padrão é direcionada para
os dispositivos mais modernos e a alternativa, chamada de
Facebook Lite\footnote{Disponível em: https://play.google.com/store/apps/details?id=com.facebook.lite}
, para dispositivos mais antigos (que utilizam a versão 2.2 do Android), que
ocupa bem menos espaço de armazenamento, é mais eficiente no uso de dados móveis,
pode ser carregada na memória do dispositivo mais rapidamente e funciona com conexões
de Internet não estáveis ou lentas. A existência de diferentes dispositivos para
os quais a plataforma Android oferece suporte é conhecido como fragmentação, e isso
se apresenta como um grande desafio para o Android, em relação a outras plataformas.
Há uma número considerável de dispositivos com versões antigas da API \cite{Gronli2014}.

De fato, customizações em larga escala de aplicações são um requisito fundamental
que vem sendo considerado por abordagens de engenharia de linhas de produtos de software
\cite{Alves2007} \cite{Alves2005}, direcionando esforços para a gerência de variabilidade
no contexto de aplicações móveis. Estudos \cite{Dehlinger2011} indicam que o desenvolvimento
de linhas de produtos de software para aplicações móveis é um desafio atual da
engenharia de software, enquanto outros enquanto outros \cite{Linares-Vasquez2013}
apontam que fazer uso de API propensas a mudanças e falhas pode ter um impacto
negativo no sucesso de aplicações. A documentação oficial do Android \cite{GuiaAndroid}
é rica em propor soluções para tais variabilidades, no entanto, a implementação
dessas soluções pode sofrer variações. Cada equipe de desenvolvimento pode tomar
decisões de projeto diferentes. Existe atualmente uma carência de estudos
específicos sobre gerência de variabilidades em aplicações para plataforma Android.

Neste contexto, este trabalho apresenta uma caracterização de como tais problemas
relacionados a gerência de variabilidade são tratadas por aplicações atuais e
qual o impacto de tais soluções no desenvolvimento de aplições reais na plataforma
Android.

\section{Problema}

Apesar da ampla adoção da plataforma Android e sua fragmentação, há uma carência
de estudos que indiquem de que forma as aplicações de tal plataforma lidam com
variabilidades relacionadas a customização da mesma para diferentes dispositivos. 

\citeonline{FalvoJunior2015} conduziu uma revisão sistemática da literatura tendo como
principal objetivo identificar estudos que integrassem os conceitos de linha de
produtos de \textit{software} (LPS) e dispositivos móveis. Após a seleção final
dos trabalhos, os autores chegaram em um conjunto de 31 artigos que foram lidos
na íntegra. Desses, 19 foram publicados em anos anteriores a 2009, portanto, antes
da criação da plataforma Android. Dos demais trabalhos, apenas um é relacionado a
plataforma Android, o qual relata uma experiência prática da implementação de uma
linha de produto de \textit{software} para monitoramento remoto de pessoas com
algum tipo de cuidado especial. Porém, tal trabalho aborda uma proposta do uso de
diagramas UML (\textit{Unified Modeling Language}) para representar variações arquiteturais.

A carência de estudos sistemáticos que identifiquem e analisem técnicas para
implementação de variabilidades ao longo das evolução de tais aplicações e da
própria plataforma Android pode trazer dificuldades para a evolução do código de
tais aplicações ou até mesmo erros durante a execução.

\section{Limitação de Trabalhos Existentes}

Vários estudos já foram conduzidos relacionando linha de produtos de software e
tecnologias móveis. \citeonline{Alves2007} propôs um método para extrair e evoluir LPSs
no nível de implementação e modelo de \textit{features}.  Nesse estudo, inicialmente
foi criado uma LPS, posteriormente evoluída com uma abordagem reativa \cite{Krueger2002}.
Em seguida, a LPS foi evoluída através de refatorações que modularizavam suas
variabilidades usando programação orientada a aspectos. \citeonline{Cirilo2012} mostram
como uma ferramenta dirigida a modelos pode ajudar na gerência de variabilidades
e instanciação automática de produtos de uma linha de produtos móveis. O trabalho
compara o impacto de duas técnicas de modularização - programação orientada a aspectos
e compilação condicional - para a derivação automática de produtos. No entanto,
tais trabalhos foram desenvolvidos no contexto da plataforma Java Micro Edition (JME) \cite{oraclejme}.

Trabalhos recentes abordam o uso da plataforma Android. Reuso de software é uma
importante motivação para desenvolvimento de linhas de produto de software, sendo
herança de classes uma das formas mais comuns. \citeonline{Mojica2014} analisam como
reuso de software ocorre nas aplicações disponíveis na Google Play. Os três principais
tipos de reuso identificados foram: (i) 18,75\% das classes herdam de uma classe
base da API Android, e 35,78\% de uma classe específica de domínio; (ii) 84,23\%
das classes aparecem em duas ou mais aplicações; (iii) 17.109 aplicações móveis
são cópias diretas de outras aplicações.

\citeonline{Laguna2011} apresentam uma LPS para monitoramento remoto de pessoas
com necessidades especiais, no entanto não são fornecidas maiores detalhes das
técnicas utilizadas para derivação de produto ou gerência de variabilidades.
\cite{Pavlic2013} relata o desenvolvimento de uma LPS capaz de produzir seis
versões de uma mesma aplicação, com diferentes funcionalidades, além de oferecer
suporte a seis idiomas. Os produtos finais são derivados utilizando as ferramentas
padrões e scripts Ant. Para gerenciamento de variabilidades foram utilizadas
mecanismos de orientação a objetos, tais como, uso de herança, especialização e
parametrização de componentes, em muitos casos através do uso de padrões de projetos.
Contudo, não são apresentados impactos do uso de tais técnicas e motivos da escolha
de uma ou outra.

\citeonline{FalvoJunior2015} apresenta uma LPS para aplicações educacionais.
A LPS e seu processo de construção, baseado em uma estratégia de adoção proativa
\cite{Krueger2002}, são detalhados no trabalho. Também é realizada uma avaliação
empírica com o objetivo de avaliar os produtos gerados. Nessa avaliação, foram
comparadas produtos gerados pela LPS e produtos criados por meio do desenvolvimento
singular de software (DSS), em que foram avaliados principalmente dois critérios:
número de defeitos e \textit{time-to-market}. Os resultados alcançados foram: (i)
o produto da LPS contém, em média, apenas um defeito, contra sete da versão DSS;
(ii) o produto da LPS é criado em menos de cinco minutos, enquanto a versão DSS
leva mais de 113. Também foram observados melhores resultados na interface gráfica
do produto LPS. Contudo, não são discutidos como as variabilidades foram
implementadas ou quais técnicas foram utilizadas.

\citeonline{Berger2014} aborda mecanismos de variabilidades em ecossistemas de
\textit{software},
incluindo
Android, Eclipse e o kernel Linux. Porém, o enfoque é na construção do ambiente e
em como tais plataformas são configuradas para atender ás demandas específicas
dos usuários, e não em suas aplicações.

\citeonline{Campos2015} realizou um estudo exploratório em doze aplicações Android
\textit{open-source}
populares com o objetivo de investigar características arquiteturais. Os três
principais aspectos investigadas foram: (i) arquitetura das aplicações;
(ii) uso de padrões de projetos; e (iii) política de tratamento de exceção.
Os principais resultados apresentados incluem: (i) estilo arquitetural MVC
(modelo-visão-controle) foi adotado pela maioria das aplicações,
mesmo que parcialmente; (ii) análise qualitativa acerca das arquiteturas encontradas;
e (iii) uma diagrama de arquitetura conceitual que sintetiza os principais achados do estudo.

Diante disso, ainda existe uma carência de estudos que abordem como as variabilidades
são implementadas em aplicações na plataforma Android, e que indiquem quais os
benefícios e pontos negativos do uso das diversas técnicas conhecidas. Em especial,
nas variabilidades oriundas de portabilidade para diferentes dispositivos e das
mudanças de bibliotecas na plataforma.

\section{Objetivos}
O objetivo geral desta dissertação de mestrado é analisar, caracterizar e comparar
técnicas de implementacao de variabilidades utilizadas por aplicações Android.

Os objetivos específicos são:
\begin{itemize}
    \item Identificar e analisar técnicas de implementação de variabilidades
        relacionadas à evolução de APIs da plataforma utilizadas por aplicações Android;
    \item Identificar e analisar as técnicas de implementação de variabilidades
        relacionadas ao suporte à dispositivos com diferentes tamanhos de tela utilizadas por aplicações Android;
    \item Identificar as principais ferramentas utilizadas para derivação automática
        de produtos utilizadas por aplicações Android.
\end{itemize}

\section{Metodologia}

Para realização dessa dissertação de mestrado, foi definido uma metodologia de trabalho
composta das seguintes etapas:
\begin{itemize}
    \item Estudo da plataforma: nessa fase foi realizado um trabalho técnico de
          estudo das tecnologias para desenvolvimento de aplicação Android, sobretudo
          procuramos observar elementos indicadores de variabilidades;
    \item Revisão bibliográfica: na revisão bibliográfica, aprofundamento o estudo
          sobre linha de produto de software, sobretudo técnicas de gerencia de
          variabilidades, e estudos que tratem desses aspectos na plataforma Android.
          Nessa fase, foi evidenciada a carência de trabalhos relacionados à gerência
          de variabilidades em aplicações Android;
    \item Estudo de análise de variabilidades de aplicações Android: essa fase tem
          por objetivo caracterizar as variabilidades em aplicações Android e
          técnicas utilizadas. Foi composta das seguintes atividades:
         \begin{itemize}
            \item Seleção das aplicações;
            \item Análise das aplicações, e;
            \item Análise dos resultados.
         \end{itemize}
\end{itemize}


\section{Organização do Documento}
O restante desse documento está organizado da seguinte forma:
\begin{itemize}
    \item O capítulo 2 apresenta os principais referenciais teóricos que serviram
            de arcabouço para o desenvolvimento deste trabalho;
    \item O capítulo 3 traz o estudo de análise de variabilidades de aplicações
          Android realizado. São apresentados resultados do estudo e discussão de tais resultados;
    \item No capítulo 4, são discutidos trabalhos relacionados;
    \item Por fim, o capítulo 5 apresenta um cronograma de atividade para continuidade da pesquisa.
\end{itemize}

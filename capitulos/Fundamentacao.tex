\chapter{Fundamentação Teórica}
Este capítulo apresenta uma visão geral sobre linha de produto de \textit{software},
gerência de variabilidades na plataforma Android, padrões de projeto e orientação
a objetos e um \textit{framework} para comparação de técnicas de implementação
de variabilidades.

\section{Linha de Produto de Software}

Linha de produto de \textit{software} é uma família de \textit{softwares} que possuem um núcleo
comum e que possibilita personalizações para cada produto específico \cite{Clements2001}.
Cada produto é criado a partir das variabilidades permitidas para a linha de produto.
Podendo ser de dois tipos: (i) variabilidades de negócio e (ii) variabilidades de
plataforma.
As primeiras são adaptações em requisitos de negócios da aplicação requeridas pelo
usuário final.
As últimas dizem respeito às variações no ambiente de hardware e software em que
uma aplicação será executada. Está intrinsecamente ligada a aspectos técnicos
relacionados ao dispositivo no qual a aplicação está instalada. Assim, uma
importante disciplina na engenharia de linha de produto de \textit{software} é a
gerencia de variabilidades.

\subsection{Gerência de Variabilidades}

Gerência de variabilidades está relacionado às atividades de suporte a variabilidades
no ciclo de vida do software, permitindo o gerenciamento de dependências e interações
entre variabilidades. Técnicas de gerência de variabilidades permitem sistematicamente
desenvolver e evoluir artefatos comuns e variáveis pertencentes a uma LPS.

\subsubsection{Variabilidades de Negócios}
Variabilidades de negócio são adaptações em requisitos de negócio da aplicação
requerida pelo usuário final. Em geral, gerentes de produto de aplicações são
responsáveis por decidir quais variabilidades de negócio uma LPS pretende oferecer
suporte. O usuário final é responsável por escolher quais variabilidades estarão
presentes nos produtos desejados. Existem diferentes soluções para implementar
variabilidades de negócio. Variabilidades alternativas que podem funcionar com
determinados produtos, por exemplo, podem ser implementadas através do uso dos
mecanismos de herança e polimorfismo de orientação a objetos, permitindo que
classes abstratas sejam estendidas para oferecer implementações específicas para
um dado produto. Padrões de projeto orientados a objetos \cite{Gamma1994} podem ser
aplicados dentro deste contexto, oferecendo implementações modularizadas para
cada uma das variantes de uma variabilidade específica. Em tais soluções, a escolha
de quais variantes serão selecionadas é delegada para uma ferramenta de derivação
de produto, a qual se responsabiliza pela inclusão das subclasses específicas
responsáveis pela inclusão no produto final.

\subsubsection{Variabilidades de Plataforma}
Plataforma refere-se ao ambiente de hardware e software em que uma aplicação irá
ser executada \cite{Preuveneers2004}. Assim, variabilidades de plataforma dizem
respeito a variações
de elementos desse conjunto. Ela está intrinsecamente ligada a aspectos técnicos
relacionados ao dispositivo no qual a aplicação é instalada, tais como:
implementações alternativas para versões que utilizam diferentes bibliotecas de classes
(APIs) e ajuste de \textit{layout} de telas para diferentes dispositivos.

\section{Metodologia}

Texto

\section{Análise dos Resultados}


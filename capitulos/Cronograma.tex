\chapter{Cronograma de Atividades}

A Tabela \ref{tab:cronograma} apresenta um cronograma de atividades para a continuidade do trabalho
durante os próximos 4 a 5 meses.

% Please add the following required packages to your document preamble:
% \usepackage[table,xcdraw]{xcolor}
% If you use beamer only pass "xcolor=table" option, i.e. \documentclass[xcolor=table]{beamer}
\begin{table}[ht]
\centering
\caption{Cronograma de atividades}
\label{tab:cronograma}
\begin{tabular}{|l|l|l|l|l|l|l|l|l|l|l|l|l|l|l|l|l|l|l|l|l|l|l|l|l|}
\hline
\textbf{Atividade}     & \multicolumn{4}{c|}{\textbf{Agosto}}                                              & \multicolumn{4}{c|}{\textbf{Setembro}}                                                                    & \multicolumn{4}{c|}{\textbf{Outubro}}                                                                     & \multicolumn{4}{c|}{\textbf{Novembro}}                                                                    & \multicolumn{4}{c|}{\textbf{Dezembro}}                                                                    & \multicolumn{4}{c|}{\textbf{Janeiro}} \\ \hline
Qualificação           &  & \cellcolor[HTML]{C0C0C0} &                          &                          &                          &                          &                          &                          &                          &                          &                          &                          &                          &                          &                          &                          &                          &                          &                          &                          &   &   &   &                           \\ \hline
Seleção das aplicações &  &                          & \cellcolor[HTML]{C0C0C0} & \cellcolor[HTML]{C0C0C0} &                          &                          &                          &                          &                          &                          &                          &                          &                          &                          &                          &                          &                          &                          &                          &                          &   &   &   &                           \\ \hline
Análise das aplicações &  &                          &                          &                          & \cellcolor[HTML]{C0C0C0} & \cellcolor[HTML]{C0C0C0} & \cellcolor[HTML]{C0C0C0} & \cellcolor[HTML]{C0C0C0} & \cellcolor[HTML]{C0C0C0} & \cellcolor[HTML]{C0C0C0} & \cellcolor[HTML]{C0C0C0} & \cellcolor[HTML]{C0C0C0} &                          &                          &                          &                          &                          &                          &                          &                          &   &   &   &                           \\ \hline
Escrita da dissertação &  &                          &                          &                          &                          &                          &                          &                          &                          &                          &                          &                          & \cellcolor[HTML]{C0C0C0} & \cellcolor[HTML]{C0C0C0} & \cellcolor[HTML]{C0C0C0} & \cellcolor[HTML]{C0C0C0} & \cellcolor[HTML]{C0C0C0} & \cellcolor[HTML]{C0C0C0} & \cellcolor[HTML]{C0C0C0} & \cellcolor[HTML]{C0C0C0} &   &   &   &                           \\ \hline
Defesa                 &  &                          &                          &                          &                          &                          &                          &                          &                          &                          &                          &                          &                          &                          &                          &                          &                          &                          &                          &                          &   &   &   & \cellcolor[HTML]{C0C0C0}  \\ \hline
\end{tabular}
\end{table}


A qualificação é a atividade atual, sendo realizada na 2ª semana de agosto.
Nas duas semanas seguintes faremos a seleção das aplicação que serão utilizadas.
A partir do aprendizado durante a fase inicial, os critérios de seleção das
aplicações foram definidos:
\begin{enumerate}
    \item Aplicação presente no F-Froid\footnote{Disponível em https://f-droid.org/},
        um catálogo de aplicações Android open-source;
    \item Versão mínima da API necessária para instalação da aplicação menor ou
        igual a 8,
        o que garante um grande intervalo de versões suportada pela aplicação;
    \item Mais de 100.000 instalações segundo o Google Play, o que garante
        popularidade à aplicação;
    \item Exclusão de aplicações projetadas específicamente para determinados
        tamanhos de telas.
\end{enumerate}

A análise das aplicações selecionadas será feita durante o mês de setembro. Nessa
etapa preliminar do trabalho, a análise foi feita de forma manual. Para a sequência,
objetivamos  tornar a análise semi ou totalmente automática. Uma ferramenta candidata
para auxiliar nessa tarefa é o Android Lint \cite{Lint}, uma ferramenta do ambiente oficial
de desenvolvimento de aplicações Android capaz de analisar o código-fonte de uma
aplicação para identificar problemas estruturais, sem a necessidade de executá-la
ou mesmo escrever testes.

Dentre as muitas verificações realizadas pelo Lint, alguns tipos são particularmente
úteis para gerencia de variabilidades. Destacamos aqui:
\begin{itemize}
    \item Traduções ausentes ou não-utilizadas
        \begin{itemize}
            \item Indicam strings da aplicação que não foram traduzidas, auxiliando
                a evitar que uma aplicação seja distribuida com tradução incompleta,
                ou conteúdo nos arquivos de strings mas que nunca serão utilizado.
        \end{itemize}
    \item Recursos não-utilizados
        \begin{itemize}
            \item Indicam sobretudo artefatos extra-código presentes no projeto
                mas que não são utilizados, auxiliando a diminuir o tamanho total das aplicações 
        \end{itemize}
    \item Inconsistencia entre elementos de configuração distintas
        \begin{itemize}
            \item Alguns elementos precisam ser definidos em vários locais, em função
                de configurações distintas dos dispositivos, sendo necessários haver
                consistência entre eles. Por exemplo, um array definido em múltiplas
                configuração precisar ter  a mesma dimensão em todas elas.
        \end{itemize}
\end{itemize}

De um total de 191 checagens do Lint, 59 estão relacionadas de alguma forma com
variabilidades.

Após análise das aplicações a versão final da dissertação será escrita durante os
meses de novembro e dezembro, para ser entregue aos membros da banca e agendarmos
a defesa para o mês de janeiro de 2017.


\chapter{Estudo de Análise de Variabilidades de Aplicações Android}

Este capítulo apresenta  o estudo realizado para análise das variabilidades de
aplicações Android. Foram analisados os códigos-fonte de 8 aplicações \textit{open-source},
com o objetivo de identificar e caracterizar as técnicas de implementação de
variabilidades utilizadas em tais aplicações.

O restante desse capítulo está organizado da seguinte maneira: a seção \ref{sec:aplicacoes}
apresenta as aplicações alvo do estudo;  a seção \ref{sec:procedimentos} descreve
os procedimentos adotados na análise; a seção \ref{sec:resultados} mostra os resultados alcançados. 


\section{Aplicações Alvo do Estudo}
\label{sec:aplicacoes}

Nessa fase preliminar, selecionamos um total de 8 aplicações \textit{open-source}
e populares, cada uma com pelo menos 10 mil downloads. Dentro desses critérios,
sete aplicações foram escolhidas de forma aleatória e então foi incluída a
Google I/O\footnote{Aplicação oficial do evento Google I/O. Repositório: https://github.com/google/iosched},
por ser uma aplicação de referência da Google. As demais são aplicações de tamanhos
e categorias variados, possuindo um grande número de usuários. São elas:
Telegram\footnote{Uma aplicação de mensagem. Repositório: https://github.com/DrKLO/Telegram},
AntennaPod\footnote{Um gestor de podcast. Repositório: https://github.com/danieloeh/AntennaPod},
Firefox\footnote{Um navegador web. Repositório: http://hg.mozilla.org/mozilla-central},
Ankidroid\footnote{Um software educativo. Repositório: https://github.com:ankidroid/Anki-Android},
K-9 Mail\footnote{Um cliente de email. Repositório: https://github.com:k9mail/k-9},
C:Geo\footnote{Uma aplicação de geocaching. Repositório: https://github.com:cgeo/cgeo}
e Zmanim\footnote{Um relógio utilizado pelos judeus. Repositório: https://bitbucket.org/jgindin/zmanim}. 

A tabela \ref{tab:aplicacoes}
apresenta maiores informações sobre tais aplicações.
A coluna ID será utilizada para identificar a aplicação em outras tabelas, se necessário.

% Please add the following required packages to your document preamble:
% \usepackage[table,xcdraw]{xcolor}
% If you use beamer only pass "xcolor=table" option, i.e. \documentclass[xcolor=table]{beamer}
\begin{table}[ht]
\centering
\caption{Aplicações analisadas no estudo}
\label{tab:aplicacoes}
\begin{tabular}{|
>{\columncolor[HTML]{FFFFFF}}l |
>{\columncolor[HTML]{FFFFFF}}l |
>{\columncolor[HTML]{FFFFFF}}l |
>{\columncolor[HTML]{FFFFFF}}l |
>{\columncolor[HTML]{FFFFFF}}l |
>{\columncolor[HTML]{FFFFFF}}l |
>{\columncolor[HTML]{FFFFFF}}l |
>{\columncolor[HTML]{FFFFFF}}l |
>{\columncolor[HTML]{FFFFFF}}l |
>{\columncolor[HTML]{FFFFFF}}l |
>{\columncolor[HTML]{FFFFFF}}l |
>{\columncolor[HTML]{FFFFFF}}l |
>{\columncolor[HTML]{FFFFFF}}l |
>{\columncolor[HTML]{FFFFFF}}l |
>{\columncolor[HTML]{FFFFFF}}l |
>{\columncolor[HTML]{FFFFFF}}l |
>{\columncolor[HTML]{FFFFFF}}l |
>{\columncolor[HTML]{FFFFFF}}l |
>{\columncolor[HTML]{FFFFFF}}l |}
\hline
\textbf{ID} & \multicolumn{4}{c|}{\cellcolor[HTML]{FFFFFF}\textbf{Aplicação}} & \multicolumn{4}{c|}{\cellcolor[HTML]{FFFFFF}\textbf{Categoria}}   & \multicolumn{4}{c|}{\cellcolor[HTML]{FFFFFF}\textbf{Downloads}}                                                   & \multicolumn{4}{c|}{\cellcolor[HTML]{FFFFFF}\textbf{\begin{tabular}[c]{@{}c@{}}Nº de linhas\\ de código\end{tabular}}} & \multicolumn{1}{c|}{\cellcolor[HTML]{FFFFFF}\textbf{\begin{tabular}[c]{@{}c@{}}API\\ Mínima\end{tabular}}} & \textbf{\begin{tabular}[c]{@{}l@{}}API\\ Alvo\end{tabular}} \\ \hline
\#1         & \multicolumn{4}{l|}{\cellcolor[HTML]{FFFFFF}Telegram}           & \multicolumn{4}{l|}{\cellcolor[HTML]{FFFFFF}Comunicação}          & \multicolumn{4}{l|}{\cellcolor[HTML]{FFFFFF}\begin{tabular}[c]{@{}l@{}}100.000.000 - \\ 500.000.000\end{tabular}} & \multicolumn{4}{l|}{\cellcolor[HTML]{FFFFFF}576 mil}                                                                   & 9                                                                                                          & 23                                                          \\ \hline
\#2         & \multicolumn{4}{l|}{\cellcolor[HTML]{FFFFFF}AntennaPod}         & \multicolumn{4}{l|}{\cellcolor[HTML]{FFFFFF}Mídia e Vídeo}        & \multicolumn{4}{l|}{\cellcolor[HTML]{FFFFFF}\begin{tabular}[c]{@{}l@{}}100.000 -\\ 500.000\end{tabular}}          & \multicolumn{4}{l|}{\cellcolor[HTML]{FFFFFF}72 mil}                                                                    & 10                                                                                                         & 23                                                          \\ \hline
\#3         & \multicolumn{4}{l|}{\cellcolor[HTML]{FFFFFF}Google I/O}         & \multicolumn{4}{l|}{\cellcolor[HTML]{FFFFFF}Livros e Referências} & \multicolumn{4}{l|}{\cellcolor[HTML]{FFFFFF}\begin{tabular}[c]{@{}l@{}}500.000 -\\ 1.000.000\end{tabular}}        & \multicolumn{4}{l|}{\cellcolor[HTML]{FFFFFF}66 mil}                                                                    & 14                                                                                                         & 22                                                          \\ \hline
\#4         & \multicolumn{4}{l|}{\cellcolor[HTML]{FFFFFF}Firefox}            & \multicolumn{4}{l|}{\cellcolor[HTML]{FFFFFF}Comunicação}          & \multicolumn{4}{l|}{\cellcolor[HTML]{FFFFFF}\begin{tabular}[c]{@{}l@{}}100.000.000 -\\ 500.000.000\end{tabular}}  & \multicolumn{4}{l|}{\cellcolor[HTML]{FFFFFF}279 mil}                                                                   & 15                                                                                                         & 22                                                          \\ \hline
\#5         & \multicolumn{4}{l|}{\cellcolor[HTML]{FFFFFF}AnkiDroid}          & \multicolumn{4}{l|}{\cellcolor[HTML]{FFFFFF}Educação}             & \multicolumn{4}{l|}{\cellcolor[HTML]{FFFFFF}\begin{tabular}[c]{@{}l@{}}1.000.000 - \\ 5.000.000\end{tabular}}     & \multicolumn{4}{l|}{\cellcolor[HTML]{FFFFFF}93 mil}                                                                    & 10                                                                                                         & 22                                                          \\ \hline
\#6         & \multicolumn{4}{l|}{\cellcolor[HTML]{FFFFFF}K-9 Mail}           & \multicolumn{4}{l|}{\cellcolor[HTML]{FFFFFF}Comunicação}          & \multicolumn{4}{l|}{\cellcolor[HTML]{FFFFFF}\begin{tabular}[c]{@{}l@{}}5.000.000 -\\ 10.000.000\end{tabular}}     & \multicolumn{4}{l|}{\cellcolor[HTML]{FFFFFF}123 mil}                                                                   & 15                                                                                                         & 22                                                          \\ \hline
\#7         & \multicolumn{4}{l|}{\cellcolor[HTML]{FFFFFF}C:Geo}              & \multicolumn{4}{l|}{\cellcolor[HTML]{FFFFFF}Entretenimento}       & \multicolumn{4}{l|}{\cellcolor[HTML]{FFFFFF}\begin{tabular}[c]{@{}l@{}}1.000.000 - \\ 5.000.000\end{tabular}}     & \multicolumn{4}{l|}{\cellcolor[HTML]{FFFFFF}149 mil}                                                                   & 9                                                                                                          & 21                                                          \\ \hline
\#8         & \multicolumn{4}{l|}{\cellcolor[HTML]{FFFFFF}Zmanim}             & \multicolumn{4}{l|}{\cellcolor[HTML]{FFFFFF}Estilo de Vida}       & \multicolumn{4}{l|}{\cellcolor[HTML]{FFFFFF}\begin{tabular}[c]{@{}l@{}}10.000 -\\ 50.000\end{tabular}}            & \multicolumn{4}{l|}{\cellcolor[HTML]{FFFFFF}48 mil}                                                                    & 10                                                                                                         & 22                                                          \\ \hline
\end{tabular}
\end{table}

Nos arquivos de configuração das aplicações, são definidas as versões mínima e
alvo da API necessária para o execução da aplicação. Se por um lado, os desenvolvedores
desejam utilizar os recursos mais modernos da plataforma, por outro desejam ter
suas aplicações instaladas na maior quantidade possível de dispositivos. Dessa
forma, a aplicação pode estar preparada para funcionar em versões antigas da
API, sem abrir mão de usar os últimas novidades. As versões mínimas encontradas
nas 8 aplicações variaram de 9 (Telegram e C:geo) a 15 (Firefox e K-9 Mail),
enquanto que apenas uma aplicação estava otimizada para a versão 21, duas
aplicações para versão mais recente (23 - Telegram e AntennadPod) e as demais
para versão 22.


\section{Procedimentos}
\label{sec:procedimentos}



\section{Resultados do Estudo}
\label{sec:resultados}

Essa seção apresenta e discute os resultados do estudo. Inicialmente, na seção 
\ref{sec:tecnicas} são discutidos as técnicas de implementação para lidar com
diferentes versçoes da API. A seção \ref{sec:mudancas} discute mudanças na API
com maior impacto no desenvolvimento das aplições.
Na seção \ref{sec:multiplas_telas} são apresentados os
resultados para o tratamento de múltiplas telas. Na seção \ref{sec:multiplos_pacotes}
é discutido o recurso provido pela plataforma de múltiplos pacotes para instalação.
Por fim, a seção \ref{sec:ferramentas} apresenta ferramentas úteis para tratar variabilidades. 

\subsection{Técnicas de Implementação de Variabilidades na API}
\label{sec:tecnicas}

\subsection{Mudanças na API com Maior Impacto no Desenvolvimento}
\label{sec:mudancas}

\subsection{Tratamento de  Múltiplas Telas}
\label{sec:multiplas_telas}

\subsection{Múltiplos Pacotes para Instalação}
\label{sec:multiplos_pacotes}

\subsection{Ferramentas Úteis para Tratar Variabilidades}
\label{sec:ferramentas}



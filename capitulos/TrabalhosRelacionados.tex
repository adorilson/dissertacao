%\textit{software}

\chapter{Trabalhos Relacionados}

Há poucos estudos abordando a análise de variabilidades no contexto de aplicações 
Android. Um deles \cite{Pavlic2013} relata uma experiência prática do desenvolvimento
de uma linha de produto de software para Android. Outro trabalho \cite{FalvoJunior2015}
propõe uma linha de produtos de software no domínio de aprendizagem móvel. No entanto,
nenhum desses trabalhos detalham aspectos de implementação e nem expõem pontos
positivos e negativos das decisões de projeto adotadas.

Contudo, vários trabalhos foram feitos abordando outras plataformas.
\citeonline{Sampaio2004} enumera os principais desafios enfrentados por uma equipe de
desenvolvimento para portar aplicações JME,a plataforma móvel mais popular da época,
para aparelhos celulares diferentes. O trabalho também analisa abordagens existentes
e propõe algumas diretivas e soluções para tais problemas. Entre esse desafios estão
as diferenças entre dispositivos, incluindo tamanho de tela, e diferentes perfis JME.
Podemos fazer um paralelo desses últimos com as diferentes versão da API do Android.

\citeonline{Garcia2006} apresentou um estudo quantitativo comparando implementações orientada a
aspectos (OA) e orientada a objetos (OO) dos padrões GoF e demostraram que implementações
OA provê separação de interesses, contudo alguns padrões podem resultar em um alto acoplamento
dos componentes, operações mais complexas e mais linhas de código que as soluções puramente OO.

Outros trabalhos realizaram uma analise comparativa de 3 estratégias de portabilidade
de jogos JME \cite{Alves2005Comparative} - abordagem incremental, transformação de
programas e pré-processamento - usando critérios como granulosidade, coesão, acoplamento,
ferramenta de suporte e manutenabilidade.

Alguns trabalhos \cite{Alves2006} focaram seus estudos sobre mecanismos de gerência
de variabilidade de artefatos que vão além de código, como imagens e sons. Demostrando
que mecanismos são esses e quais as razões de escolha baseados em critérios como
binding-time, performance e reusabilidade.

Nesse trabalho, a análise e detecção dos padrões de projeto foi feito de forma manual,
mas existem propostas para detecção automática \cite{Rasool2011}, embora não
necessariamente aplicados a Android. Outros se propõe a detectar anti-padrões em
aplicações Android \cite{Hecht2015}.

\citeonline{LinaresVasquez2014} apresenta estudo preliminares demonstrando evidências sobre o
impacto de mudanças na API do Android sobre as aplicações. Eles acreditam que
identificar mudanças e bugs na API, além de ajudar a resolver muitos problemas na
aplicações, também ajudarão a manter uma boa avaliação dos usuários. Por outro lado,
devido a problemas causados nas aplicações em função de evolução na API, os
desenvolvedores tem evitado uso de APIs instáveis, conforme apontado por \citeonline{McDonnell2013}.

